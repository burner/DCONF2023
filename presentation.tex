\documentclass[aspectratio=169,notes]{beamer}
\usepackage{lmodern}
\usepackage{adjustbox}
\usepackage[T1]{fontenc}
\usepackage{textcomp}
\usepackage{animate}
\usepackage{underscore}
\usepackage{pdfpc-commands}
\usepackage{xmpmulti}
\usepackage{multimedia}
\usepackage{epstopdf}
\usepackage{bbding}
\usepackage[absolute,overlay]{textpos}
\usepackage[most]{tcolorbox}
%\usepackage{ctex}

\definecolor{Title}{rgb}{0.94,0.52,0.08}
\setbeamercolor{frametitle}{bg=Title,fg=black}

% footnote without number
\makeatletter
\def\blfootnote{\xdef\@thefnmark{}\@footnotetext}
\makeatother

\usepackage{hyperref}
\usepackage{scalerel}
\def\thumbup{\scalerel*{\includegraphics{thumbup.png}}{O}}
\usepackage{listings}
\lstdefinelanguage{D}
{
  % list of keywords
  morekeywords={ abstract, alias, align, asm, assert, auto, body, bool, break,
	byte, case, cast, catch, cdouble, cent, cfloat, char, class, const,
	continue, default, double, else, enum, export, extern, false, final, finally,
	float, for, foreach, foreach_reverse, function, goto, idouble, if, ifloat,
	immutable, import, in, inout, int, interface, invariant, ireal, is, lazy,
	long, mixin, module, new, nothrow, null, out, override, package, pragma,
	private, protected, public, pure, real, ref, return, scope, shared, short,
	static, string, struct, super, switch, synchronized, template, this, throw,
	true, try, typeid, typeof, ubyte, ucent, uint, ulong, union,
	unittest, delegate, @safe, @property
	ushort, version, void, wchar, while, with, __FILE__, __FILE_FULL_PATH__,
	__MODULE__, __LINE__, __FUNCTION__, __PRETTY_FUNCTION__, __gshared,
	__traits, __vector, __parameters
  },
  otherkeywords= { @property, @safe },
  sensitive=false, % keywords are not case-sensitive
  morecomment=[l]{//}, % l is for line comment
  morecomment=[s]{/*}{*/}, % s is for start and end delimiter
  morecomment=[s]{/+}{+/}, % s is for start and end delimiter
  morestring=[b]{"}, % defines that strings are enclosed in double quotes
  morestring=[b]{`} % defines that strings are enclosed in double quotes
}
\usepackage{color}
\definecolor{eclipseBlue}{RGB}{42,0.0,255}
\definecolor{eclipseGreen}{RGB}{63,127,95}
\definecolor{eclipsePurple}{RGB}{127,0,85}

% Set Language
\lstset{
  language={D},
  basicstyle=\small\ttfamily, % Global Code Style
  captionpos=b, % Position of the Caption (t for top, b for bottom)
  extendedchars=true, % Allows 256 instead of 128 ASCII characters
  tabsize=2, % number of spaces indented when discovering a tab 
  columns=fixed, % make all characters equal width
  keepspaces=true, % does not ignore spaces to fit width, convert tabs to spaces
  showstringspaces=false, % lets spaces in strings appear as real spaces
  breaklines=true, % wrap lines if they don't fit
  numbers=left, % show line numbers at the left
  numberstyle=\tiny\ttfamily, % style of the line numbers
  commentstyle=\color{eclipseGreen}, % style of comments
  keywordstyle=\color{eclipsePurple}, % style of keywords
  stringstyle=\color{eclipseBlue}, % style of strings
}
\definecolor{lightgray}{rgb}{.9,.9,.9}
\definecolor{darkgray}{rgb}{.4,.4,.4}
\definecolor{purple}{rgb}{0.65, 0.12, 0.82}
\lstdefinelanguage{TypeScript}{
	keywords={break, case, catch, continue, debugger, default, delete, do, else,
		false, from, finally, for, function, if, in, instanceof, new, null, return, switch,
		this, throw, true, try, typeof, var, void, while, with, interface,
		class, export, boolean, throw, implements, import, this, const, let,
		of, =>},
	morecomment=[l]{//},
	morecomment=[s]{/*}{*/},
	morestring=[b]',
	morestring=[b]",
	ndkeywords={},
	keywordstyle=\color{blue}\bfseries,
	ndkeywordstyle=\color{darkgray}\bfseries,
	identifierstyle=\color{black},
	commentstyle=\color{purple}\ttfamily,
	stringstyle=\color{red}\ttfamily,
	sensitive=true
}

\colorlet{punct}{red!60!black}
\definecolor{background}{HTML}{EEEEEE}
\definecolor{delim}{RGB}{20,105,176}
\colorlet{numb}{magenta!60!black}

\lstdefinelanguage{GraphQL}{
    basicstyle=\normalfont\ttfamily,
    numbers=left,
    stepnumber=1,
    showstringspaces=false,
    breaklines=true,
	keywords={type, schema, mutation, subscription, __type, __schema, kind,
		on, fragment, query},
    literate=
     *{0}{{{\color{numb}0}}}{1}
      {1}{{{\color{numb}1}}}{1}
      {2}{{{\color{numb}2}}}{1}
      {3}{{{\color{numb}3}}}{1}
      {4}{{{\color{numb}4}}}{1}
      {5}{{{\color{numb}5}}}{1}
      {6}{{{\color{numb}6}}}{1}
      {7}{{{\color{numb}7}}}{1}
      {8}{{{\color{numb}8}}}{1}
      {9}{{{\color{numb}9}}}{1}
      {:}{{{\color{punct}{:}}}}{1}
      {,}{{{\color{punct}{,}}}}{1}
      {\{}{{{\color{delim}{\{}}}}{1}
      {\}}{{{\color{delim}{\}}}}}{1}
      {[}{{{\color{delim}{[}}}}{1}
      {]}{{{\color{delim}{]}}}}{1},
}

\lstdefinelanguage{json}{
    basicstyle=\normalfont\ttfamily,
    numbers=left,
    stepnumber=1,
    showstringspaces=false,
    breaklines=true,
    literate=
     *{0}{{{\color{numb}0}}}{1}
      {1}{{{\color{numb}1}}}{1}
      {2}{{{\color{numb}2}}}{1}
      {3}{{{\color{numb}3}}}{1}
      {4}{{{\color{numb}4}}}{1}
      {5}{{{\color{numb}5}}}{1}
      {6}{{{\color{numb}6}}}{1}
      {7}{{{\color{numb}7}}}{1}
      {8}{{{\color{numb}8}}}{1}
      {9}{{{\color{numb}9}}}{1}
      {:}{{{\color{punct}{:}}}}{1}
      {,}{{{\color{punct}{,}}}}{1}
      {\{}{{{\color{delim}{\{}}}}{1}
      {\}}{{{\color{delim}{\}}}}}{1}
      {[}{{{\color{delim}{[}}}}{1}
      {]}{{{\color{delim}{]}}}}{1},
}
\usepackage{tikz}
\usetikzlibrary{shadows,calc}
\usepackage{xkeyval}
\usepackage{todonotes}
\presetkeys{todonotes}{inline}{}
\defbeamertemplate{description item}{align left}{\insertdescriptionitem\hfill}
\usetheme{metropolis}					 % Use metropolis theme
\usepackage[
    backend=biber,
	sorting=none,
    url=true 
]{biblatex}
\addbibresource{biblio.bib}
\setbeamertemplate{bibliography item}{\insertbiblabel}

\title{Simple @safe D --- How to make enemies quickly}
\date{DConf 2023}
\author{Dr.\,Robert Schadek}

\begin{document}
	\maketitle

	\begin{frame}[fragile]{The Problem 1/2}
		\lstinputlisting[language=D,firstnumber=1,firstline=3,lastline=16,basicstyle=\footnotesize\ttfamily]{code.d}
	\end{frame}

	\begin{frame}[fragile]{The Problem 2/2}
		\begin{itemize}
			\item Just from the syntax DIP1000 	
			\item DIP1021 -- Argument Ownership and Function Calls	
			\item DIP1035 -- \lstinline@@system@ Variables ... are dead on arrival	
		\end{itemize}
		\begin{itemize}	
			\item Thinking that D is the C/C++ successor ... it is not, that is rust
			\item Thinking \lstinline|@safe| languages are the new thing ...
they are not. Most languages are safe already, python, JS, java
		\end{itemize}
	\end{frame}

	\section{Hopefully, there are still some people in the room with me at this point}

	\begin{frame}[fragile]{The Solution}
		\begin{itemize}
			\item old school \lstinline|@safe|
			\item No unary \lstinline|&| --- remove this from the grammar in \lstinline|@safe|
			\item No \lstinline@return@ by \lstinline@ref@
		\end{itemize}
	\end{frame}

	\begin{frame}[fragile]{The Consequences}
		\begin{itemize}
			\item No need for DIP1000, DIP1021, and DIP 1035
			\item No user defined \lstinline|@safe| container that behave like in-builds
			\item No Manual Memory Management (MMM) in \lstinline|@safe| code
			\item Clear definition of \lstinline|@property|
		\end{itemize}
	\end{frame}

	\begin{frame}[fragile]{DIP1000}
		\begin{columns}[T]
		\begin{column}{0.49\textwidth}
		\lstinputlisting[language=D,firstnumber=1,firstline=20,lastline=40,basicstyle=\scriptsize\ttfamily]{code.d}
		\end{column}
		\begin{column}{0.49\textwidth}
		\lstinputlisting[language=D,firstnumber=1,firstline=41,lastline=50,basicstyle=\scriptsize\ttfamily]{code.d}
		\end{column}
		\end{columns}
	\end{frame}

	\section{Continuations}

	\begin{frame}[t]
		\frametitle{\lstinline@assert@s}
		\begin{columns}[T]
		\begin{column}{0.49\textwidth}
		\lstinputlisting[language=D,firstnumber=1,firstline=1,lastline=17,basicstyle=\scriptsize\ttfamily]{asserttest.d}
		\end{column}
		\begin{column}{0.49\textwidth}
		\lstinputlisting[language=D,firstnumber=19,firstline=19,basicstyle=\scriptsize\ttfamily]{asserttest.d}
		\begin{itemize}
			\item \lstinline@dmd -release -run asserttest.d@
			\pause
			\item No assert, in/out contrast, or invariant
		\end{itemize}
		\end{column}
		\end{columns}
	\end{frame}

	\begin{frame}[t]
		\frametitle{Scott Meyers}
		\begin{quote}
		The last thing D needs is somebody like me
		\end{quote}
	\end{frame}

	\begin{frame}[t]
		\frametitle{Template Constraints}
		\lstinputlisting[language=D,firstnumber=1,firstline=1,lastline=14,basicstyle=\scriptsize\ttfamily]{templateconstraints.d}
	\end{frame}

	\begin{frame}[t]
		\frametitle{Template Constraints continued}
		\lstinputlisting[language=D,firstnumber=15,firstline=15,lastline=29,basicstyle=\scriptsize\ttfamily]{templateconstraints.d}
	\end{frame}

	\begin{frame}[t]
		\frametitle{Template Constraints continued}
		\lstinputlisting[language=D,firstnumber=30,firstline=30,lastline=43,basicstyle=\scriptsize\ttfamily]{templateconstraints.d}
	\end{frame}

	\begin{frame}[t]
		\frametitle{Template Constraints less terrible}
		\lstinputlisting[language=D,firstnumber=45,firstline=45,lastline=61,basicstyle=\scriptsize\ttfamily]{templateconstraints.d}
	\end{frame}

	\begin{frame}[t]
		\frametitle{Nested Functions}
		\lstinputlisting[language=D,firstnumber=1,firstline=1,lastline=25,basicstyle=\scriptsize\ttfamily]{nestedfunction.d}
			
	\end{frame}

	\begin{frame}[t]
		\frametitle{Nested Imports}
		\lstinputlisting[language=D,firstnumber=1,firstline=1,lastline=25,basicstyle=\scriptsize\ttfamily]{nestedimports.d}
	\end{frame}

	\section{Please don't add}

	\begin{frame}[t]
		\frametitle{build-in un-named tuple}
	\end{frame}

\end{document}
